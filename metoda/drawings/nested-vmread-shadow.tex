% Copyright (c) 2016 Grigory Rechistov <grigory.rechistov@gmail.com>
% This work is licensed under the Creative Commons Attribution-NonCommercial-ShareAlike 4.0 Worldwide.
% To view a copy of this license, visit http://creativecommons.org/licenses/by-nc-sa/4.0/.


% This file allows to produce either a separate PDF/PNG image
% See standalone documentation to understand underlying magic

\documentclass[tikz,convert={density=150,size=600,outext=.png}]{standalone}
\usetikzlibrary{shapes, calc, arrows, fit, positioning, decorations, patterns, decorations.pathreplacing, chains, snakes}
% Copyright (c) 2016 Grigory Rechistov <grigory.rechistov@gmail.com>
% This work is licensed under the Creative Commons Attribution-NonCommercial-ShareAlike 4.0 Worldwide.
% To view a copy of this license, visit http://creativecommons.org/licenses/by-nc-sa/4.0/.

\usepackage{fontspec}
\usepackage{xunicode} % some extra unicode support
\usepackage{xltxtra}

\usepackage{amsfonts}
\usepackage{amsmath}
\usepackage{longtable}
\usepackage{csquotes}

\usepackage{polyglossia}
\setdefaultlanguage[spelling=modern]{russian} % for polyglossia
\setotherlanguage{english} % for polyglossia

% Common settings for all fonts
% 1. Attempt to make fonts be of the same size
% 2. Support TeX ligatures like — = emdash, << >> = guillemets
\defaultfontfeatures{Scale=MatchLowercase, Mapping=tex-text}

% Use Computer Modern Unicode
\newfontfamily\russianfont{CMU Serif}
\setromanfont{CMU Serif}
\setsansfont{CMU Sans Serif}
\setmonofont{CMU Typewriter Text}

% Copyright (c) 2016 Grigory Rechistov <grigory.rechistov@gmail.com>
% This work is licensed under the Creative Commons Attribution-NonCommercial-ShareAlike 4.0 Worldwide.
% To view a copy of this license, visit http://creativecommons.org/licenses/by-nc-sa/4.0/.

% Common packages, commands and their configuration

\newcommand{\abbr}{\textit{англ.}\ }
\newcommand{\todo}[1][]{\textcolor{red}{TODO #1}}

\usepackage{graphicx}
\graphicspath{{pictures/}} % path to pictures, trailing slash is mandatory.

\usepackage{hyperref}
\hypersetup{colorlinks=true, linkcolor=black, filecolor=black, citecolor=black, urlcolor=black , pdfauthor=Grigory Rechistov <grigory.rechistov@gmail.com>, pdftitle=Программное моделирование вычислительных систем}

\usepackage{footnpag}
\usepackage{indentfirst}
\usepackage{underscore}
\usepackage{url}

\usepackage{listings}
\lstset{basicstyle=\footnotesize\ttfamily, breaklines=true, keepspaces=true }

\usepackage{tikz}
\usetikzlibrary{shapes, calc, arrows, fit, positioning, decorations, patterns, decorations.pathreplacing, chains, snakes}
\usepackage{bytefield}

\usepackage{pgfplots} % Draw plots inside TeX
\pgfplotsset{compat=1.10}

\usepackage{standalone} % Clever way to build TikZ pictures either to PDF or to PNG


\graphicspath{{../pictures/}} % path to pictures, trailing slash is mandatory.

% The actual drawing follows
\begin{document}
\begin{tikzpicture}[>=latex, font=\small, node distance=0.2cm]

\draw[->] (0,-0.5) -- (10,-0.5) node [pos=0.9, below] {Время};

\draw (0,0) rectangle (3, 1);
\draw (0,1.2) rectangle (3, 2.2);

\draw (4,0) rectangle   (8, 1);
\draw (4,1.2) rectangle (8, 2.2);
\draw (4,2.4) rectangle (8, 3.4);

\begin{scope}[font=\tiny, text height=0.12cm, every node/.style={fill=white}, inner xsep=0pt]
    \draw[->] (0.3, 1.7) -- +(0.1, -1.2) node[pos=0, above] {vmexit} coordinate[pos=1] (v1);
    \draw[->] (2.6, 0.5) -- +(0.1, 1.2) node[pos=0, below] {vmresume} coordinate[pos=0] (v2);
    \path (v1) -- (v2) node[pos=0.33] {vmread} node[pos=0.67] {vmwrite};

    \draw[->] (4.3, 3.0) -- +(0.1, -2.5) node[pos=0, above] {vmexit} coordinate[pos=1] (v3);
    \draw[->] (7.6, 0.5) -- +(0.1, 2.5) node[pos=0, below] {vmresume} coordinate[pos=0] (v4);

    \draw[->] (4.5, 0.5) -- +(0.1, 1.2) coordinate[pos=1] (v5);
    \draw[->] (7.3, 1.7) -- +(0.1, -1.2) coordinate[pos=0] (v6);


    \path (v5) -- (v6) node[pos=0.2] {vmread} node[pos=0.6] {vmwrite} node[pos=0.95] {vmresume};
    
    \path (v3) -- (v4) ;
\end{scope}

\node[align=center] at (1.5, 4) {Один уровень\\виртуализации};
\node[align=center] at (7, 4) {Два уровня\\виртуализации, теневая VMCS};

\node[] at (-0.5, 0.5) {L0};
\node[] at (-0.5, 1.7) {L1};
\node[] at (-0.5, 2.8) {L2};
\end{tikzpicture}


\end{document}
