% Copyright (c) 2016 Grigory Rechistov <grigory.rechistov@gmail.com>
% This work is licensed under the Creative Commons Attribution-NonCommercial-ShareAlike 4.0 Worldwide.
% To view a copy of this license, visit http://creativecommons.org/licenses/by-nc-sa/4.0/.

\section{\Questions к главе \ref{alternatives}} %\label{alternatives-questions}

\subsection*{Вариант 1}

\begin{questions}

\question[3] Сформулируйте закон Литтла.
\begin{solution}[2cm]
Среднее число $N$ клиентов за достаточно долгосрочный период в устойчиво функционирующей системе  равно средней норме или скорости прибытия, умноженной на определённое за тот же период среднее время $T$, которое один клиент проводит в системе: $N = \lambda T$
\end{solution}

\question[3] Сформулируйте закон использования.
\begin{solution}[2cm]
$U = X \cdot S $
\end{solution}

\question[3] Сформулируйте закон вынужденного потока.
\begin{solution}[2cm]
Пропорциональность загруженности подсистем полной пропускной способности всей системы при условии ненасыщенного состояния отдельных центров: $X_k = V_k X$.
\end{solution}

\question[3] Какая из методик изучения опирается на генерацию случайных внешних воздействий:
\begin{choices}
\choice изучение сетей обслуживания,
\correctchoice метод Монте-Карло,
\choice функциональная симуляция,
\choice потактовая симуляция?
\end{choices}


\end{questions}

\subsection*{Вариант 2}

\begin{questions}

\question[3] Сформулируйте закон баланса потока.
\begin{solution}[1cm]
Скорость прибытия клиентов равна пропускной способности системы: $\lambda = X$.
\end{solution}

\question[3] Сформулируйте соотношение для времени отклика системы.
\begin{solution}[1cm]
$R = N/X - Z$
\end{solution}

\question[3] Сформулируйте соотношение для времени отклика системы.
\begin{solution}[1cm]
$R = N/X - Z$
\end{solution}

\question[3] Какие требования  выдвигаются генератору случайных чисел при их использовании в симуляторе:
\begin{choices}
\correctchoice случайность и взаимная независимость,
\choice возможность изменять функцию распределения генерируемой последовательности,
\correctchoice высокая скорость генерации случайной последовательности,
\choice максимально возможная ширина выдаваемых чисел,
\choice криптографическая стойкость создаваемой последовательности?
\end{choices}

\end{questions}

% К каждой лекции должно быть от 8 до 12 задач, у каждой задачи должно быть 3-5 вариантов формулировок примерно одинаковой сложности. Допускается объединение нескольких последовательных лекций в одну тему и подготовка тестов к темам.
% Задачи должны полностью соответствовать материалам лекций, то есть лекциях должно быть достаточно информации для ответа на все вопросы.
% Формулировка каждого варианта задачи должна содержать всю необходимую информацию и не должна ссылаться на тексты внутри лекции, картинки или другие задачи или варианты задачи.
% Правильные ответы выделяются знаком «+» перед их формулировкой. Правильных ответов может быть несколько. Для тестов с несколькими ответами как минимум один ответ должен быть правильным и как минимум один ответ должен быть неправильным. 
% 
% Структура теста к лекции
% 
% \subsection*{Задача 1}
% 
% \paragraph{Вариант 1} 
% 
%     Чему равно 2+2?
%         Ответ 1. 3
%         + Ответ 2. 4
%         …
%         Ответ N. 5
% \paragraph{Вариант 2}
%     Чему равно 2*2?
%         + Ответ 1. 4
%         + Ответ 2. 2+2
%         …
%         Ответ N. 5
% \paragraph{Вариант 3}
% 
%     Чему равно 2-2?
%         Ответ 1. 0
% 
% 
%         
% \section{Просто подборка вопросов}
% 


 
 